% \begin{longtable}[]{@{}llllll@{}}
\begin{table}
\begin{tabularx}{\textwidth}{|c|c|c|X|c|c|}
\toprule
obs\_id & string & obs & unique earth view observation identifier. &
&\tabularnewline
obs\_time\_tai93 & double & obs & earth view observation midtime for
each FOV & seconds since 1993-01-01 00:00 & bnds\tabularnewline
obs\_time\_utc & uint16 & obs, utc\_tuple & UTC earth view observation
time as an array of integers: year, month, day, hour, minute, second,
millisec, microsec & &\tabularnewline
lat & float & obs & latitude of FOV center & degrees\_north &
bnds\tabularnewline
lon & float & obs & longitude of FOV center & degrees\_east &
bnds\tabularnewline
land\_frac & float & obs & land fraction over the FOV & unitless
&\tabularnewline
surf\_alt & float & obs & mean surface altitude wrt earth model over the
FOV & m &\tabularnewline
surf\_alt\_sdev & float & obs & standard deviation of surface altitude
within the FOV & m &\tabularnewline
sun\_glint\_lat & float & obs & sun glint spot latitude at
scan\_mid\_time. Fill for night observations. & degrees\_north
&\tabularnewline
sun\_glint\_lon & float & obs & sun glint spot longitude at
scan\_mid\_time. Fill for night observations. & degrees\_east
&\tabularnewline
sol\_zen & float & obs & solar zenith angle at the center of the FOV &
degree &\tabularnewline
sol\_azi & float & obs & solar azimuth angle at the center of the FOV
(clockwise from North) & degree &\tabularnewline
sun\_glint\_dist & float & obs & Distance from the center of the
calculated sun glint spot to the center of the spot. Note that there may
not be a glint for cloudy or land cases and in ocean cases the glint can
move based on wind conditions. Fill for night observations. & m
&\tabularnewline
\bottomrule
\end{tabularx}
\end{table}
