
\subsection{Attributes}
\label{attrs}

\begin{center}
\begin{xltabular}{1.1\textwidth}{|m{4cm}|l|l|
>{\hsize=12\hsize\linewidth=\hsize}X|
>{\hsize=18\hsize\linewidth=\hsize}X|
}

\hline
Name & Type & Size & Value & Description\tabularnewline\hline
\hline
keywords & string & 1 & EARTH SCIENCE \textgreater{}
SPECTRAL/ENGINEERING \textgreater{} INFRARED WAVELENGTHS \textgreater{}
INFRARED RADIANCE & A comma-separated list of key words and/or phrases.
Keywords may be common words or phrases, terms from a controlled
vocabulary (GCMD is often used), or URIs for terms from a controlled
vocabulary (see also "keywords\_vocabulary" attribute).\tabularnewline\hline
Conventions & string & 1 & CF-1.6\textbackslash, ACDD-1.3 & A
comma-separated list of the conventions that are followed by the
dataset.\tabularnewline\hline
history & string & 1 & & Provides an audit trail for modifications to
the original data. This attribute is also in the NetCDF Users Guide:
'This is a character array with a line for each invocation of a program
that has modified the dataset. Well-behaved generic netCDF applications
should append a line containing: date, time of day, user name, program
name and command arguments.' To include a more complete description you
can append a reference to an ISO Lineage entity; see NOAA EDM ISO
Lineage guidance.\tabularnewline\hline
source & string & 1 & AIRS and CrIS instrument telemetry & The method of
production of the original data. If it was model-generated, source
should name the model and its version. If it is observational, source
should characterize it. This attribute is defined in the CF Conventions.
Examples: 'temperature from CTD \#1234'; 'world model
v.0.1'.\tabularnewline\hline
processing\_level & string & 1 & 1 & A textual description of the
processing (or quality control) level of the data.\tabularnewline\hline
product\_name\_type\_id & string & 1 & L1 & Product name as it appears
in product\_name (L1A, L1B, L2, SNO\_AIRS\_CrIS)\tabularnewline\hline
comment & string & 1 & & Miscellaneous information about the data or
methods used to produce it. Can be empty.\tabularnewline\hline
acknowledgment & string & 1 & Support for this research was provided by
NASA. & A place to acknowledge various types of support for the project
that produced this data.\tabularnewline\hline
license & string & 1 & Limited to Sounder SIPS affiliates & Provide the
URL to a standard or specific license, enter "Freely Distributed" or
"None", or describe any restrictions to data access and distribution in
free text.\tabularnewline\hline
standard\_name\_vocabulary & string & 1 & CF Standard Name Table v28 &
The name and version of the controlled vocabulary from which variable
standard names are taken. (Values for any standard\_name attribute must
come from the CF Standard Names vocabulary for the data file or product
to comply with CF.) Example: 'CF Standard Name Table
v27'.\tabularnewline\hline
date\_created & string & 1 & Unassigned & The date on which this version
of the data was created. (Modification of values implies a new version,
hence this would be assigned the date of the most recent values
modification.) Metadata changes are not considered when assigning the
date\_created. The ISO 8601:2004 extended date format is recommended, as
described in the Attribute Content Guidance section.\tabularnewline\hline
creator\_name & string & 1 & Unassigned & The name of the person (or
other creator type specified by the creator\_type attribute) principally
responsible for creating this data.\tabularnewline\hline
creator\_email & string & 1 & Unassigned & The email address of the
person (or other creator type specified by the creator\_type attribute)
principally responsible for creating this data.\tabularnewline\hline
creator\_url & string & 1 & Unassigned & The URL of the person (or other
creator type specified by the creator\_type attribute) principally
responsible for creating this data.\tabularnewline\hline
institution & string & 1 & Unassigned & Processing facility that
produced this file\tabularnewline\hline
project & string & 1 & Sounder SIPS & The name of the project(s)
principally responsible for originating this data. Multiple projects can
be separated by commas, as described under Attribute Content Guidelines.
Examples: 'PATMOS-X', 'Extended Continental Shelf
Project'.\tabularnewline\hline
product\_name\_project & string & 1 & SNDR & The name of the project as
it appears in the file name. 'SNDR' for all Sounder SIPS products, even
AIRS products.\tabularnewline\hline
publisher\_name & string & 1 & Unassigned & The name of the person (or
other entity specified by the publisher\_type attribute) responsible for
publishing the data file or product to users, with its current metadata
and format.\tabularnewline\hline
publisher\_email & string & 1 & Unassigned & The email address of the
person (or other entity specified by the publisher\_type attribute)
responsible for publishing the data file or product to users, with its
current metadata and format.\tabularnewline\hline
publisher\_url & string & 1 & Unassigned & The URL of the person (or
other entity specified by the publisher\_type attribute) responsible for
publishing the data file or product to users, with its current metadata
and format.\tabularnewline\hline
geospatial\_bounds & string & 1 & & Describes the data's 2D or 3D
geospatial extent in OGC's Well-Known Text (WKT) Geometry format
(reference the OGC Simple Feature Access (SFA) specification). The
meaning and order of values for each point's coordinates depends on the
coordinate reference system (CRS). The ACDD default is 2D geometry in
the EPSG:4326 coordinate reference system. The default may be overridden
with geospatial\_bounds\_crs and geospatial\_bounds\_vertical\_crs (see
those attributes). EPSG:4326 coordinate values are longitude (decimal
degrees\_east) and latitude (decimal degrees\_north), in that order.
Longitude values in the default case are limited to the {[}-180, 180)
range. Example: 'POLYGON ((-111.29 40.26, -111.29 41.26, -110.29 41.26,
-110.29 40.26, -111.29 40.26))'.\tabularnewline\hline
geospatial\_bounds\_crs & string & 1 & EPSG:4326 & The coordinate
reference system (CRS) of the point coordinates in the
geospatial\_bounds attribute. This CRS may be 2-dimensional or
3-dimensional, but together with geospatial\_bounds\_vertical\_crs, if
that attribute is supplied, must match the dimensionality, order, and
meaning of point coordinate values in the geospatial\_bounds attribute.
If geospatial\_bounds\_vertical\_crs is also present then this attribute
must only specify a 2D CRS. EPSG CRSs are strongly recommended. If this
attribute is not specified, the CRS is assumed to be EPSG:4326.
Examples: 'EPSG:4979' (the 3D WGS84 CRS), 'EPSG:4047'.\tabularnewline\hline
geospatial\_lat\_min & float & 1 & 9.9692099683868690e+36f & Describes a
simple lower latitude limit; may be part of a 2- or 3-dimensional
bounding region. Geospatial\_lat\_min specifies the southernmost
latitude covered by the dataset.\tabularnewline\hline
geospatial\_lat\_max & float & 1 & 9.9692099683868690e+36f & Describes a
simple upper latitude limit; may be part of a 2- or 3-dimensional
bounding region. Geospatial\_lat\_max specifies the northernmost
latitude covered by the dataset.\tabularnewline\hline
geospatial\_lon\_min & float & 1 & 9.9692099683868690e+36f & Describes a
simple longitude limit; may be part of a 2- or 3-dimensional bounding
region. geospatial\_lon\_min specifies the westernmost longitude covered
by the dataset. See also geospatial\_lon\_max.\tabularnewline\hline
geospatial\_lon\_max & float & 1 & 9.9692099683868690e+36f & Describes a
simple longitude limit; may be part of a 2- or 3-dimensional bounding
region. geospatial\_lon\_max specifies the easternmost longitude covered
by the dataset. Cases where geospatial\_lon\_min is greater than
geospatial\_lon\_max indicate the bounding box extends from
geospatial\_lon\_max, through the longitude range discontinuity meridian
(either the antimeridian for -180:180 values, or Prime Meridian for
0:360 values), to geospatial\_lon\_min; for example,
geospatial\_lon\_min=170 and geospatial\_lon\_max=-175 incorporates 15
degrees of longitude (ranges 170 to 180 and -180 to
-175).\tabularnewline\hline
time\_coverage\_start & string & 1 & & Nominal start time. Describes the
time of the first data point in the data set. Use the ISO 8601:2004 date
format, preferably the extended format as recommended in the Attribute
Content Guidance section.\tabularnewline\hline
time\_of\_first\_valid\_obs & string & 1 & & Describes the time of the
first valid data point in the data set. Use the ISO 8601:2004 date
extended format.\tabularnewline\hline
time\_coverage\_mid & string & 1 & & Describes the midpoint between the
nominal start and end times. Use the ISO 8601:2004 date format,
preferably the extended format as recommended in the Attribute Content
Guidance section.\tabularnewline\hline
time\_coverage\_end & string & 1 & & Nominal end time. Describes the
time of the last data point in the data set. Use ISO 8601:2004 date
format, preferably the extended format as recommended in the Attribute
Content Guidance section.\tabularnewline\hline
time\_of\_last\_valid\_obs & string & 1 & & Describes the time of the
last valid data point in the data set. Use the ISO 8601:2004 date
extended format.\tabularnewline\hline
time\_coverage\_duration & string & 1 & P0000-00-00T00:06:00 & Describes
the duration of the data set. Use ISO 8601:2004 duration format,
preferably the extended format as recommended in the Attribute Content
Guidance section.\tabularnewline\hline
product\_name\_duration & string & 1 & m06 & Product duration as it
appears in product\_name (m06 means six minutes)\tabularnewline\hline
creator\_type & string & 1 & institution & Specifies type of creator
with one of the following: 'person', 'group', 'institution', or
'position'. If this attribute is not specified, the creator is assumed
to be a person.\tabularnewline\hline
creator\_institution & string & 1 & Jet Propulsion Laboratory -\/-
California Institute of Technology & The institution of the creator;
should uniquely identify the creator's institution. This attribute's
value should be specified even if it matches the value of
publisher\_institution, or if creator\_type is
institution.\tabularnewline\hline
product\_version & string & 1 & vxx.xx.xx & Version identifier of the
data file or product as assigned by the data creator. For example, a new
algorithm or methodology could result in a new
product\_version.\tabularnewline\hline
keywords\_vocabulary & string & 1 & GCMD:GCMD Keywords & If you are
using a controlled vocabulary for the words/phrases in your "keywords"
attribute, this is the unique name or identifier of the vocabulary from
which keywords are taken. If more than one keyword vocabulary is used,
each may be presented with a prefix and a following comma, so that
keywords may optionally be prefixed with the controlled vocabulary key.
Example: 'GCMD:GCMD Keywords, CF:NetCDF COARDS Climate and Forecast
Standard Names'.\tabularnewline\hline
platform & string & 1 & JPSS-1 \textgreater{} Joint Polar Satellite
System - 1\textbackslash, SUOMI-NPP \textgreater{} Suomi National
Polar-orbiting Partnership\textbackslash, AQUA \textgreater{} Earth
Observing System & Name of the platform(s) that supported the sensor
data used to create this data set or product. Platforms can be of any
type, including satellite, ship, station, aircraft or other. Indicate
controlled vocabulary used in platform\_vocabulary.\tabularnewline\hline
platform\_vocabulary & string & 1 & GCMD:GCMD Keywords & Controlled
vocabulary for the names used in the "platform"
attribute.\tabularnewline\hline
product\_name\_platform & string & 1 & SS1330 & Platform name as it
appears in product\_name\tabularnewline\hline
instrument & string & 1 & AIRS \textgreater{} Atmospheric Infrared
Sounder\textbackslash, CrIS \textgreater{} Cross-track Infrared Sounder
& Name of the contributing instrument(s) or sensor(s) used to create
this data set or product. Indicate controlled vocabulary used in
instrument\_vocabulary.\tabularnewline\hline
instrument\_vocabulary & string & 1 & GCMD:GCMD Keywords & Controlled
vocabulary for the names used in the "instrument"
attribute.\tabularnewline\hline
product\_name\_instr & string & 1 & CHIRP & Instrument name as it
appears in product\_name\tabularnewline\hline
product\_name & string & 1 & & Canonical fully qualified product name
(official file name)\tabularnewline\hline
product\_name\_variant & string & 1 & std & Processing variant
identifier as it appears in product\_name. 'std' (shorthand for
'standard') is to be the default and should be what is seen in all
public products.\tabularnewline\hline
product\_name\_version & string & 1 & vxx\_xx\_xx & Version number as it
appears in product\_name (v01\_00\_00)\tabularnewline\hline
product\_name\_producer & string & 1 & T & Production facility as it
appears in product\_name (single character) 'T' is the default, for
unofficial local test products\tabularnewline\hline
product\_name\_timestamp & string & 1 & yymmddhhmmss & Processing
timestamp as it appears in product\_name (yymmddhhmmss)\tabularnewline\hline
product\_name\_extension & string & 1 & nc & File extension as it
appears in product\_name (typically nc)\tabularnewline\hline
granule\_number & ushort & 1 & & granule number of day
(1-240)\tabularnewline\hline
product\_name\_granule\_number & string & 1 & g000 & zero-padded string
for granule number of day (g001-g240)\tabularnewline\hline
gran\_id & string & 1 & yyyymmddThhmm & Unique granule identifier
yyyymmddThhmm of granule start, including year, month, day, hour, and
minute of granule start time\tabularnewline\hline
geospatial\_lat\_mid & float & 1 & 9.9692099683868690e+36f & granule
center latitude\tabularnewline\hline
geospatial\_lon\_mid & float & 1 & 9.9692099683868690e+36f & granule
center longitude\tabularnewline\hline
featureType & string & 1 & trajectory & structure of data in
file\tabularnewline\hline
data\_structure & string & 1 & trajectory & a character string
indicating the internal organization of the data with currently allowed
values of 'grid', 'station', 'trajectory', or 'swath'. The 'structure'
here generally describes the horizontal structure and in all cases data
may also be functions, for example, of a vertical coordinate and/or
time. (If using CMOR pass this in a call to
cmor\_set\_cur\_dataset\_attribute.)\tabularnewline\hline
cdm\_data\_type & string & 1 & Trajectory & The data type, as derived
from Unidata's Common Data Model Scientific Data types and understood by
THREDDS. (This is a THREDDS "dataType", and is different from the CF
NetCDF attribute 'featureType', which indicates a Discrete Sampling
Geometry file in CF.)\tabularnewline\hline
id & string & 1 & Unassigned & An identifier for the data set, provided
by and unique within its naming authority. The combination of the
"naming authority" and the "id" should be globally unique, but the id
can be globally unique by itself also. IDs can be URLs, URNs, DOIs,
meaningful text strings, a local key, or any other unique string of
characters. The id should not include white space
characters.\tabularnewline\hline
naming\_authority & string & 1 & Unassigned & The organization that
provides the initial id (see above) for the dataset. The naming
authority should be uniquely specified by this attribute. We recommend
using reverse-DNS naming for the naming authority; URIs are also
acceptable. Example: 'edu.ucar.unidata'.\tabularnewline\hline
identifier\_product\_doi & string & 1 & Unassigned & digital
signature\tabularnewline\hline
identifier\_product\_doi\_authority & string & 1 & Unassigned & digital
signature source\tabularnewline\hline
algorithm\_version & string & 1 & & The version of the algorithm in
whatever format is selected by the developers. After the main algorithm
name and version, versions from multiple sub-algorithms may be
concatenated with semicolon separators. (ex: 'CCAST 4.2; BB emis from
MIT 2016-04-01') Must be updated with every delivery that changes
numerical results.\tabularnewline\hline
production\_host & string & 1 & & Identifying information about the host
computer for this run. (Output of linux "uname -a"
command.)\tabularnewline\hline
format\_version & string & 1 & v02.02.07 & Format
version.\tabularnewline\hline
input\_file\_names & string & 1 & & Semicolon-separated list of names or
unique identifiers of files that were used to make this product. There
will always be one space after each semicolon. There is no final
semicolon.\tabularnewline\hline
input\_file\_types & string & 1 & & Semicolon-separated list of tags
giving the role of each input file in input\_file\_names. There will
always be one space after each semicolon. There is no final
semicolon.\tabularnewline\hline
input\_file\_dates & string & 1 & & Semicolon-separated list of creation
dates for each input file in input\_file\_names. There will always be
one space after each semicolon. There is no final
semicolon.\tabularnewline\hline
orbitDirection & string & 1 & & Orbit is ascending and/or descending.
Values are "Ascending" or "Descending" if the entire granule fits that
description. "NorthPole" and "SouthPole" are used for polar-crossing
granules. "NA" is used when a determination cannot be
made.\tabularnewline\hline
day\_night\_flag & string & 1 & & Data is day or night. "Day" means
subsatellite point for all valid scans has solar zenith angle less than
90 degrees. "Night" means subsatellite point for all valid scans has
solar zenith angle greater than 90 degrees. "Both" means the dataset
contains valid observations with solar zenith angle above and below 90
degrees. "NA" means a value could not be determined.\tabularnewline\hline
AutomaticQualityFlag & string & 1 & Missing & "Passed": all spectra are
present and calibrated with no quality issues; "Suspect": at least one
spectrum is missing or calibrated with quality issues; "Failed": no
calibrated spectra; "Missing": no downlinked data.\tabularnewline\hline
AutomaticQualityFlagExplanation & string & 1 & 'Passed': all spectra are
present and calibrated with no quality issues; 'Suspect': at least one
spectrum is missing or calibrated with quality issues; 'Failed': no
calibrated spectra; 'Missing': no downlinked data. & A text explanation
of the criteria used to set AutomaticQualityFlag; including thresholds
or other criteria.\tabularnewline\hline
qa\_pct\_data\_missing & float & 1 & & Percentage of expected
observations that are missing.\tabularnewline\hline
qa\_pct\_data\_geo & float & 1 & & Percentage of expected observations
that are successfully geolocated.\tabularnewline\hline
qa\_pct\_data\_sci\_mode & float & 1 & & Percentage of expected
observations that were taken while the instrument was in science mode
and are successfully geolocated.\tabularnewline\hline
qa\_no\_data & string & 1 & TRUE & A simple indicator of whether this is
an "empty" granule with no data from the instrument. "TRUE" or
"FALSE".\tabularnewline\hline
title & string & 1 & 13:30 orbit L1 CHIRP & a succinct description of
what is in the dataset. (= ECS long name)\tabularnewline\hline
summary & string & 1 & The CHIRP Level 1 product for the 13:30
sun-synchronous orbit consists of calibrated radiance spectra at a
common resolution derived from hyperspectral instruments on
EOS-Aqua\textbackslash, S-NPP\textbackslash, and JPSS-1/NOAA-20
platforms adjusted to form a continuous climate-quality record. & A
paragraph describing the dataset, analogous to an abstract for a
paper.\tabularnewline\hline
shortname & string & 1 & SSYN1330CHIRP1\_placeholder & ECS Short
Name\tabularnewline\hline
product\_group & string & 1 & l1\_chirp & The group name to be used for
this product when it is collected in a multi-group file type, like SNO
or calsub\tabularnewline\hline
metadata\_link & string & 1 & http://disc.sci.gsfc.nasa.gov/ & A URL
that gives the location of more complete metadata. A persistent URL is
recommended for this attribute.\tabularnewline\hline
references & string & 1 & & ATDB and design documents describing
processing algorithms. Can be empty.\tabularnewline\hline
contributor\_name & string & 1 & UMBC Atmospheric Spectroscopy
Laboratory: Larrabee Strow & The names of any individuals or
institutions that contributed to the creation of this
data.\tabularnewline\hline
contributor\_role & string & 1 & CrIS L1B Scientist & The roles of any
individuals or institutions that contributed to the creation of this
data.\tabularnewline\hline
wnum\_delta\_lw & float & 1 & 0.625f & Difference between adjacent
wavenumbers in longwave spectrum, in cm-1\tabularnewline\hline
wnum\_delta\_mw & float & 1 & 0.83333333333f & Difference between
adjacent wavenumbers in midwave spectrum, in cm-1\tabularnewline\hline
wnum\_delta\_sw & float & 1 & 1.25f & Difference between adjacent
wavenumbers in shortwave spectrum, in cm-1\tabularnewline\hline

\end{xltabular}
\end{center}

